% !TeX root = RJwrapper.tex
\title{Introduction to General Swaption Valution with SabrSwaption}
\author{by Terry Leitch, Author Two}

\maketitle

\abstract{%
An abstract of less than 150 words.
}

\subsection{Introduction}\label{introduction}

RQuantlib is a package that exposes some of the functionality of
Quantlib to users via a simple R interface. There is a broad set of
functions that include fixed income and interest vrate derivatives. This
article goes in depth regarding a new function made available that
exposes Peter Casper's SABR modeling and the more recent work on path
dependent Bermudan swaptions via his Markov Functional model
\citet{caspers}.

\subsection{Interest Option Modeling}\label{interest-option-modeling}

Interest rates present numerous complex issues for a modeler. The
underlying instruments that options are written on are themselves
nonlinear, which compounds the complexity of the valuation. In addition,
the underlying instruments are quoted on numerous different compounding
and daycount basis and the underlying option prices do not fit any of
the standard random innovation methods, so modeling takes significant
work and care. Fortunately the Quantlib library handles all of these
complexities.

The most signifcant of these is the lack of fit to standard innovation
models. I order to address this, more complex models are used to fit the
market prices add random variables and parameters to fit the observed
market prices. A significant issue with this is that a fit is required
at each exiration for every tenor, so to reproduce the market prices,
numerous local dynamic model are generated as opposed to a single global
model.

European options, which are path indepnedent, are fit with these models
with little issue as long as the option lines up with a current
expiration, but as tme evolves they fall in between the expirations, so
an interpolation is required. Quantlib's modeling for SABR takes this
into account and prices by interpolating on epiration and strike.

Path dependent options are more difficault as they cut through not only
stike and expirations, but also tenors. This is what the Markov
Functional model addresses for Bermudan pricing.

Both of these are made available in R for swaption modeling through the
SabrSwaption function.

This section may contain a figure such as Figure \ref{figure:rlogo}.

\begin{figure}[htbp]
  \centering
  \includegraphics{Rlogo}
  \caption{The logo of R.}
  \label{figure:rlogo}
\end{figure}

\subsection{Another section}\label{another-section}

There will likely be several sections, perhaps including code snippets,
such as:

\begin{Schunk}
\begin{Sinput}
x <- 1:10
x
\end{Sinput}
\begin{Soutput}
#>  [1]  1  2  3  4  5  6  7  8  9 10
\end{Soutput}
\end{Schunk}

\subsection{Summary}\label{summary}

This file is only a basic article template. For full details of
\emph{The R Journal} style and information on how to prepare your
article for submission, see the
\href{https://journal.r-project.org/share/author-guide.pdf}{Instructions
for Authors}. \bibliography{RJreferences}

\address{%
Terry Leitch\\
Ruxton Advisors\\
line 1\\ line 2\\
}
\href{mailto:tleitch1@jhu.edu}{\nolinkurl{tleitch1@jhu.edu}}

\address{%
Author Two\\
Affiliation\\
line 1\\ line 2\\
}
\href{mailto:author2@work}{\nolinkurl{author2@work}}

