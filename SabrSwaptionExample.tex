% !TeX root = RJwrapper.tex
\title{Introduction to General Swaption Valution with SabrSwaption}
\author{by Terry Leitch, Author Two}

\maketitle

\abstract{%
An abstract of less than 150 words.
}

\subsection{Introduction}\label{introduction}

RQuantlib is a package that exposes some of the functionality of
Quantlib to users via a simple R interface. There is a broad set of
functions that include fixed income and interest vrate derivatives. This
article goes in depth regarding a new function made available that
exposes Peter Casper's SABR modeling and the more recent work on path
dependent Bermudan swaptions via his Markov Functional model
\citet{caspers}.

Recent functions have been added to the RQuantlib package that expose
more relevant interest rate option functionaility to current swaption
market. One function, discussed in greater detail here, is SabrSwaption,
which provides access to the European and Bermudan exercise swap
options, or swaptions.

The underlying math that is required to line up no-arbitrage pricing of
swaptions is varied in approach and the methodologies implemented in
Quantlib are illustrated in \citet{Brigo} for European exercise while
the methodology for the Bermudan is documented in \citiet{Caspers}. The
complication anyone implementing these models is the exponential grwoth
in data requirements. While a yield curve can be described eficiently
with 10-20 data points in one dimension, the volatility data is a cube.
Given the size of the data an example data set is provided with the
package in order to help the user get started with the new function. The
yield curve provided, tsQuotes, is a small dataset, while the volatility
data in the vcube dataset is 1384 observations for a single currency.
Handling the scale of this complexity requires appropriate tools, and a
Shiny app is included to help the user get started with inquiring and
managing the scale of the data.

\subsection{Interest Option Modeling}\label{interest-option-modeling}

An example is included in the

This section may contain a figure such as Figure \ref{figure:rlogo}.

\begin{figure}[htbp]
  \centering
  \includegraphics{Rlogo}
  \caption{The logo of R.}
  \label{figure:rlogo}
\end{figure}

\subsection{Another section}\label{another-section}

There will likely be several sections, perhaps including code snippets,
such as:

\begin{Schunk}
\begin{Sinput}
x <- 1:10
x
\end{Sinput}
\begin{Soutput}
#>  [1]  1  2  3  4  5  6  7  8  9 10
\end{Soutput}
\end{Schunk}

\subsection{Summary}\label{summary}

This file is only a basic article template. For full details of
\emph{The R Journal} style and information on how to prepare your
article for submission, see the
\href{https://journal.r-project.org/share/author-guide.pdf}{Instructions
for Authors}. \bibliography{RJreferences}

\address{%
Terry Leitch\\
Ruxton Advisors\\
line 1\\ line 2\\
}
\href{mailto:tleitch1@jhu.edu}{\nolinkurl{tleitch1@jhu.edu}}

\address{%
Author Two\\
Affiliation\\
line 1\\ line 2\\
}
\href{mailto:author2@work}{\nolinkurl{author2@work}}

